\documentclass[12pt]{article}
\usepackage{amsmath}
\usepackage{hyperref}
\usepackage[onehalfspacing]{setspace}
\usepackage[top=2.5cm, bottom=2.5cm, left=2.5cm, right=2.5cm]{geometry}
\usepackage{mathptmx}
\usepackage{enumitem}
\usepackage{listings}
\usepackage{minted}
\usepackage{dirtree}
\hypersetup{
      colorlinks=true,
}
\setlist{nosep}

\title{A functional programming language\\Maturaarbeit}
\author{Enea Giger\\Betreuungsperson: Adrian Lüthi}
\date{Gymnasium Burgdorf\\\today}


\begin{document}
\maketitle
\newpage

\begin{abstract}
      I describe the creation of a new functional programming language called Imp
      heavily based on the language ML
\end{abstract}

\tableofcontents
\newpage


\section{Introduction}

\section{Fundamentals}
To understand how this programming language is different, it is important to understand how programming languages work in general.

\subsection{The Lexer}
The first step of a programming language implementation is almost always the lexer.
It is responsible for splitting up the source code into tokens, little chunks like the name of a variable or a statement.
This can most often be done in one single loop over the code.
In the implementation of Imp, this is the case.
These tokens can then be used as input for the parser.

\subsection{The Parser}
The Parser of a programming language is responsible for ensuring the syntax is correct and, in most cases,
building an Abstract Syntax Tree (AST) out of the tokens.
This can be done in many ways, though what methods are applicable
depend on the syntax of the language.
\subsubsection{Pratt Parsing}
Pratt Parsing \cite{prattTopOperatorPrecedence1973}
is the method that is used for the expressions in this language.
It works with a list of rules for what to do when a certain type of token is encountered
either at the start or in the middle of an expression.
When parsing an expression using this method,
it starts with the right-binding power of 0.

Then the first type of token is looked up
in the list of rules for the start
(in the null denotation, as it's called in the paper).
This results in a function which is then called
which parses the expression starting with this token.
For example, if this is a negation, it also parses the argument
by recursively calling the expression function again.
Once this prefix is parsed the algorithm enters a loop:
\begin{enumerate}
      \item \label{start}
            Lookup the next token in the in- and postfix table.
            (This is called the left denotation in the paper.)
            This results in a left-binding power and a function
      \item If there is no function listed or
            the left-binding power stored is less or equal
            to the right-binding power, return the current expression
      \item Otherwise, call the parsing function listed with the current expression
            and store the result back in the current expression
      \item Return to step \ref{start}.
\end{enumerate}
For infix operators like multiplication, this parsing function
will call the expression function again, but with a higher right-binding power
so as to ensure for example that addition can't be parsed in a product,
since the left binding power of addition is less than
the right-binding power of multiplication,
which is the way precedence is solved by this algorithm.
By changing the numbers, it can also be regulated whether the operator
is calculated left-to-right or right-to-left.

\subsection{Type Checking/Inference}
After the program has been parsed, there traditionally follows
a semantic analysis step. This step most often consists of a type checker
(if the language is typed) and an optimizing step.

The Type Checking step ensures that all programs are well-typed,
meaning only operations are applied on the arguments, that can
be done, so no runtime type error can occur as it does in dynamically typed
languages like Python.

\subsubsection{Hindley-Milner type inference}
The Hindley-Milner type inference, as described in
\cite{damasPrincipalTypeschemesFunctional1982}
is originally a way to determine types for functions in the
polymorphic lambda calculus with an addition of let expressions.

\paragraph{Polymorphic Lambda Calculus}
The Polymorphic Lambda Calculus is a simple model for computation.
The only operations that are possible are:
\begin{itemize}
      \item $\lambda x.\:e$,
            which creates a function where $x$ is the name of the argument
            and $e$ is an expression that is given as a result
      \item $x$, where $x$ is the name of a variable
            and the result is the value of that variable
      \item $e_1\:e_2$, where both $e_1$ and $e_2$ are expressions and the result
            is the result of the call of function $e_1$ with argument $e_2$\item $c$, where $c$ is a constant
      \item $let\:x = e_1\:in\:e_2$, where x is the name of the variable
            which is defined to be the result of expression $e_1$ in the scope of expression $e_2$\end{itemize}

\subsubsection{The Inference Algorithm}
The Hindley-Milner type inference algorithm computes the types of expressions
by essentially first assuming a maximally generic type and refining it step by step
using information like the operations used (e.g. function calls).
If there turns out to be no type possible, the program is not well-typed.
Important for the type system is the definition, that only let-defined variables
can be polymorphic (i.e. can take on multiple different types).

\subsection{The Optimizer}
The next step in the semantic analysis step is in most languages an optimizing step.
This is done in many passes over the code in most languages. The type of optimizations
possible depend on how the program is executed, whether the language is typed,
sometimes also depending on the target machine and many other aspects.

\subsection{The Interpreter}
The interpreter is the next step in many programming languages.
It is a program intended to run the user-written program in the new
language. There are many types of interpreters:
\begin{itemize}
      \item \textbf{Byte-Code Interpreters}, which run a
            a custom-defined intermediate representation that is similar to how
            machine code works, but more specific to the language. Most interpreted languages
            use one, because it is most often the fastest possibility.
      \item \textbf{Threaded Interpreters}, where there is no byte sequence, but pointers
            to functions to be called or the next instruction. This has the problem that there
            are many indirect calls and pointer indirections being done, reducing the speed.
      \item \textbf{Abstract-Syntax-Tree Interpreters}, which directly evaluate the tree,
            which has more information available, but might not be optimal because of syntax
            nodes which need to be skipped and there are more indirections than
            \textbf{Byte-Code Interpreters}
\end{itemize}
Both the byte-code and threaded interpreters require a compiler.

\subsection{The Compiler}
The Compiler uses the parsed program and converts it into another langugage,
most often the machine language. Programs using the machine language are most often
faster than ones run by interpreted languages, thanks to directly being able to use
optimizations in the CPU, which can be used since the program is specific
and not general like an interpreter.

Many compilers nowadays don't directly convert the code to machine code,
but depend on a backend. One of the most popular backends is \textbf{LLVM}.

\subsubsection{LLVM}
\textbf{LLVM}, which originally stood for Low Level Virtual Machine, is
a backend capable of converting a custom language called
LLVM IR (Intermediate Representation) into code for several targets,
like x86-based cpus (most Intel and AMD chips), ARM-based cpus, but also
WASM (Web Assembly, a language understood by browsers).

It is coded in C++, though bindings exist for several languages, most notably
C, which can be used from many programming languages.

\section{Implementation}

\subsection{Used Software}

The Software was developed using the Editors Visual Studio Code and Zed.
Zed was used primarily, but VS Code was very useful to debug the code.
Once Zed got Debugger support however, it was the only editor used.

The Programming Language itself was implemented in Zig, a new low-level programming
language that aims to improve on C. Zig was used in this project, since it has builtin
support for tagged unions and a build system. However, for the compiler, C was still used for
a few builtin functions.

For the ability to syntax highlight code in Imp, a parser was made using tree-sitter.
This parser is written in JavaScript. Since the grammar of the language is too complex,
a custom lexer was also needed, which itself was written in C.
To state what parts of the code should be colored how, Queries had to be written as well,
which are written in an unnamed language unique to Tree-sitter.

Highlighting with Tree-sitter was chosen specifically since it is used
in the editor Zed, enabling a custom extension to be written for Imp.

This article itself was written in LaTex, while the listings are generated
by \href{https://github.com/TomLebeda/chroma_code}{chromacode}.


\subsection{The Language}
\begin{lstlisting}[
	% there are many more options of styling, see the official documentation, these are just the defaults I like
	frame=single, % make single-line frame around the verbatim
	framesep=2mm, % put some more spacing between the frame and text
	aboveskip=5mm, % put some more space above the box
	basicstyle={\linespread{0.9}\small\ttfamily}, % use typewriter (monospace) font
	caption={Declaration of the type Either}, % set the caption text
	captionpos=b, % put the caption at the bottom (b) or top (t) or both (bt)
    label={listing:either}, % label to be referenced via \ref{}
	numbers=left, % line numbers on the left
	numberstyle={\scriptsize\ttfamily\color{black!60}}, % the style for line numbers
	escapeinside={<@}{@>} % between those sequences are commands evaluated
]
<@\textcolor[HTML]{3010CF}{\textit{\textbf{\texttt{type}}}}@><@\textcolor[HTML]{000000}{\texttt{\ }}@><@\textcolor[HTML]{1F8F42}{\texttt{Either}}@><@\textcolor[HTML]{000000}{\texttt{\ }}@><@\textcolor[HTML]{1F8F42}{\texttt{a}}@><@\textcolor[HTML]{000000}{\texttt{\ }}@><@\textcolor[HTML]{1F8F42}{\texttt{b}}@><@\textcolor[HTML]{000000}{\texttt{\ }}@><@\textcolor[HTML]{1041FF}{\texttt{=}}@><@\textcolor[HTML]{000000}{\texttt{\ }}@><@\textcolor[HTML]{255CFF}{\texttt{Left}}@><@\textcolor[HTML]{000000}{\texttt{\ }}@><@\textcolor[HTML]{1F8F42}{\texttt{a}}@><@\textcolor[HTML]{000000}{\texttt{\ |\ }}@><@\textcolor[HTML]{255CFF}{\texttt{Right}}@><@\textcolor[HTML]{000000}{\texttt{\ }}@><@\textcolor[HTML]{1F8F42}{\texttt{b}}@>

\end{lstlisting}
\begin{lstlisting}[
	% there are many more options of styling, see the official documentation, these are just the defaults I like
	frame=single, % make single-line frame around the verbatim
	framesep=2mm, % put some more spacing between the frame and text
	aboveskip=5mm, % put some more space above the box
	basicstyle={\linespread{0.9}\small\ttfamily}, % use typewriter (monospace) font
	caption={Definition of the fold left function}, % set the caption text
	captionpos=b, % put the caption at the bottom (b) or top (t) or both (bt)
    label={listing:foldl}, % label to be referenced via \ref{}
	numbers=left, % line numbers on the left
	numberstyle={\scriptsize\ttfamily\color{black!60}}, % the style for line numbers
	escapeinside={<@}{@>} % between those sequences are commands evaluated
]
<@\textcolor[HTML]{000000}{\texttt{\#\ Compile\ the\ values\ of\ a\ List\ using\ a\ given\ function}}@>
<@\textcolor[HTML]{000000}{\texttt{}}@><@\textcolor[HTML]{3010CF}{\textit{\textbf{\texttt{let}}}}@><@\textcolor[HTML]{000000}{\texttt{\ }}@><@\textcolor[HTML]{255CFF}{\texttt{foldl}}@><@\textcolor[HTML]{000000}{\texttt{:\ }}@><@\textcolor[HTML]{1F8F42}{\texttt{(b\ ->\ a\ ->\ b}}@><@\textcolor[HTML]{FF1010}{\texttt{)}}@><@\textcolor[HTML]{1F8F42}{\texttt{\ ->\ b\ ->\ List\ a\ ->\ b}}@><@\textcolor[HTML]{000000}{\texttt{\ }}@><@\textcolor[HTML]{1041FF}{\texttt{=}}@>
<@\textcolor[HTML]{000000}{\texttt{\ \ \ \ }}@><@\textcolor[HTML]{3010CF}{\textit{\textbf{\texttt{lambda}}}}@><@\textcolor[HTML]{000000}{\texttt{\ }}@><@\textcolor[HTML]{EA5110}{\textit{\texttt{conj}}}@><@\textcolor[HTML]{1041FF}{\texttt{.}}@><@\textcolor[HTML]{000000}{\texttt{\ }}@><@\textcolor[HTML]{3010CF}{\textit{\textbf{\texttt{lambda}}}}@><@\textcolor[HTML]{000000}{\texttt{\ }}@><@\textcolor[HTML]{EA5110}{\textit{\texttt{acc}}}@><@\textcolor[HTML]{1041FF}{\texttt{.}}@><@\textcolor[HTML]{000000}{\texttt{\ }}@><@\textcolor[HTML]{3010CF}{\textit{\textbf{\texttt{lambda}}}}@><@\textcolor[HTML]{000000}{\texttt{\ }}@><@\textcolor[HTML]{EA5110}{\textit{\texttt{xs}}}@><@\textcolor[HTML]{1041FF}{\texttt{.}}@>
<@\textcolor[HTML]{000000}{\texttt{\ \ \ \ \ \ \ \ }}@><@\textcolor[HTML]{3010CF}{\textit{\textbf{\texttt{case}}}}@><@\textcolor[HTML]{000000}{\texttt{\ }}@><@\textcolor[HTML]{EA5110}{\textit{\texttt{xs}}}@><@\textcolor[HTML]{000000}{\texttt{\ }}@><@\textcolor[HTML]{3010CF}{\textit{\textbf{\texttt{of}}}}@>
<@\textcolor[HTML]{000000}{\texttt{\ \ \ \ \ \ \ \ \ \ \ \ }}@><@\textcolor[HTML]{000000}{\texttt{Nil}}@><@\textcolor[HTML]{000000}{\texttt{\ =>\ }}@><@\textcolor[HTML]{EA5110}{\textit{\texttt{acc}}}@>
<@\textcolor[HTML]{000000}{\texttt{\ \ \ \ \ \ \ \ \ \ \ \ |\ }}@><@\textcolor[HTML]{255CFF}{\texttt{Cons}}@><@\textcolor[HTML]{000000}{\texttt{\ }}@><@\textcolor[HTML]{000000}{\texttt{x}}@><@\textcolor[HTML]{000000}{\texttt{\ }}@><@\textcolor[HTML]{000000}{\texttt{rest}}@><@\textcolor[HTML]{000000}{\texttt{\ =>}}@>
<@\textcolor[HTML]{000000}{\texttt{}}@><@\textcolor[HTML]{000000}{\texttt{\ \ \ \ \ \ \ \ \ \ \ \ \ \ \ \ }}@><@\textcolor[HTML]{000000}{\texttt{foldl}}@><@\textcolor[HTML]{000000}{\texttt{\ }}@><@\textcolor[HTML]{EA5110}{\textit{\texttt{conj}}}@><@\textcolor[HTML]{000000}{\texttt{\ }}@><@\textcolor[HTML]{FF1010}{\texttt{(}}@><@\textcolor[HTML]{EA5110}{\textit{\texttt{conj}}}@><@\textcolor[HTML]{000000}{\texttt{\ }}@><@\textcolor[HTML]{EA5110}{\textit{\texttt{acc}}}@><@\textcolor[HTML]{000000}{\texttt{\ }}@><@\textcolor[HTML]{000000}{\texttt{x}}@><@\textcolor[HTML]{FF1010}{\texttt{)}}@><@\textcolor[HTML]{000000}{\texttt{\ }}@><@\textcolor[HTML]{000000}{\texttt{rest}}@>

\end{lstlisting}

The Syntax of the language consists of statements and expressions.

There are only two types of statements.
\begin{itemize}
      \item Type statements, which define an algebraic data type,
            meaning a type whose values can be constructed by various
            constructors when given specific types.
            These types can be generic.
            An example can be seen in Listing \ref{listing:either}, where the type Either is defined,
            which can be constructed by either a value of type a or of type b.
      \item Let statements, which define an (immutable) variable
            in the rest of the file. If the variable is assigned a
            lambda expression, it can be recursive
\end{itemize}
Statements are separated from expressions by not being indented,
which has the consequence that all expressions have to be indented,
but makes it so that there does not need to be an end keyword
for these statements.
This is the only case of significant whitespace in the language.

The types of expressions that exist are:
\begin{itemize}
      \item Lambda expressions, which work exactly
            the same as in pure lambda calculus
      \item Let expressions, which work the same as in pure
            lambda calculus, except, as with the statement,
            it can be recursive if assigned a lambda
      \item Variables,
      \item Function Calls,
      \item Constants, meaning Integers (whole numbers), Floats,
            One-Byte Unicode Characters, and Booleans,
      \item If expressions,
      \item Operator applications (both prefix and infix),
      \item Case expressions, which match a value of an ADT
            based on the Constructor,
      \item and syntax sugar for lists and strings,
            which are lists of characters
\end{itemize}

Comments start with a $\#$ sign and end with the end of the line.

An example of code in this language is the fold left function
given in Listing \ref{listing:foldl}.

\subsection{The Lexer}
The lexer of Imp is a simple single-pass lexer, that computes
the tokens on-demand. This is done to avoid additional memory
allocations other than creating the AST.

\subsubsection{Token Types}
\begin{itemize}
      \item The Keywords $let$ and $in$, $lambda$, $if$, $then$ and $else$,
            $case$ and $of$, $or$, and $and$,
      \item The special characters $.$, $=$, $:$, $($ and $)$, $[$ and $]$
      \item The special sequences $->$ and $=>$,
      \item all operators for comparison and mathematical operations,
            and the special operator $;$,
      \item Numbers, Booleans, Characters and Strings,
      \item Identifiers,
      \item and Special Tokens for new lines and the end of the file
\end{itemize}

\subsection{The Parser}
\begin{lstlisting}[
	% there are many more options of styling, see the official documentation, these are just the defaults I like
	frame=single, % make single-line frame around the verbatim
	framesep=2mm, % put some more spacing between the frame and text
	aboveskip=5mm, % put some more space above the box
	basicstyle={\linespread{0.9}\small\ttfamily}, % use typewriter (monospace) font
	caption={The main loop for parsing the expressions}, % set the caption text
	captionpos=b, % put the caption at the bottom (b) or top (t) or both (bt)
    label={listing:expressions}, % label to be referenced via \ref{}
	numbers=left, % line numbers on the left
	numberstyle={\scriptsize\ttfamily\color{black!60}}, % the style for line numbers
	escapeinside={<@}{@>} % between those sequences are commands evaluated
]
<@\textcolor[HTML]{3010CF}{\textit{\textbf{\texttt{pub}}}}@><@\textcolor[HTML]{000000}{\texttt{\ }}@><@\textcolor[HTML]{3010CF}{\textbf{\texttt{fn}}}@><@\textcolor[HTML]{000000}{\texttt{\ }}@><@\textcolor[HTML]{255CFF}{\texttt{expression}}@><@\textcolor[HTML]{FF1010}{\texttt{(}}@><@\textcolor[HTML]{EA5110}{\texttt{self}}@><@\textcolor[HTML]{1041FF}{\texttt{:}}@><@\textcolor[HTML]{000000}{\texttt{\ }}@><@\textcolor[HTML]{1041FF}{\texttt{*}}@><@\textcolor[HTML]{EA5110}{\texttt{Parser}}@><@\textcolor[HTML]{1041FF}{\texttt{,}}@><@\textcolor[HTML]{000000}{\texttt{\ }}@><@\textcolor[HTML]{EA5110}{\texttt{rbp}}@><@\textcolor[HTML]{1041FF}{\texttt{:}}@><@\textcolor[HTML]{000000}{\texttt{\ }}@><@\textcolor[HTML]{EA5110}{\texttt{Precedence}}@><@\textcolor[HTML]{FF1010}{\texttt{)}}@><@\textcolor[HTML]{000000}{\texttt{\ }}@><@\textcolor[HTML]{1041FF}{\texttt{!}}@><@\textcolor[HTML]{1041FF}{\texttt{*}}@><@\textcolor[HTML]{EA5110}{\texttt{AST}}@><@\textcolor[HTML]{000000}{\texttt{\ }}@><@\textcolor[HTML]{FF1010}{\texttt{\{}}@>
<@\textcolor[HTML]{000000}{\texttt{\ \ \ \ }}@><@\textcolor[HTML]{7777BB}{\textit{\texttt{//\ nud\ stands\ for\ null\ denotation}}}@>
<@\textcolor[HTML]{000000}{\texttt{\ \ \ \ }}@><@\textcolor[HTML]{3010CF}{\textit{\textbf{\texttt{const}}}}@><@\textcolor[HTML]{000000}{\texttt{\ }}@><@\textcolor[HTML]{EA5110}{\texttt{nud}}@><@\textcolor[HTML]{000000}{\texttt{\ }}@><@\textcolor[HTML]{1041FF}{\texttt{=}}@><@\textcolor[HTML]{000000}{\texttt{\ }}@><@\textcolor[HTML]{EA5110}{\texttt{self}}@><@\textcolor[HTML]{1041FF}{\texttt{.}}@><@\textcolor[HTML]{255CFF}{\texttt{getNud}}@><@\textcolor[HTML]{FF1010}{\texttt{(}}@><@\textcolor[HTML]{FF1010}{\texttt{)}}@><@\textcolor[HTML]{1041FF}{\texttt{;}}@>
<@\textcolor[HTML]{000000}{\texttt{\ \ \ \ }}@><@\textcolor[HTML]{3010CF}{\textit{\textbf{\texttt{const}}}}@><@\textcolor[HTML]{000000}{\texttt{\ }}@><@\textcolor[HTML]{EA5110}{\texttt{n}}@><@\textcolor[HTML]{000000}{\texttt{\ }}@><@\textcolor[HTML]{1041FF}{\texttt{=}}@><@\textcolor[HTML]{000000}{\texttt{\ }}@><@\textcolor[HTML]{EA5110}{\texttt{nud}}@><@\textcolor[HTML]{000000}{\texttt{\ }}@><@\textcolor[HTML]{1010FF}{\texttt{orelse}}@><@\textcolor[HTML]{000000}{\texttt{\ }}@><@\textcolor[HTML]{FF1010}{\texttt{\{}}@>
<@\textcolor[HTML]{000000}{\texttt{\ \ \ \ \ \ \ \ }}@><@\textcolor[HTML]{911010}{\textit{\texttt{try}}}@><@\textcolor[HTML]{000000}{\texttt{\ }}@><@\textcolor[HTML]{EA5110}{\texttt{self}}@><@\textcolor[HTML]{1041FF}{\texttt{.}}@><@\textcolor[HTML]{000000}{\texttt{errs}}@><@\textcolor[HTML]{1041FF}{\texttt{.}}@><@\textcolor[HTML]{255CFF}{\texttt{errorAt}}@><@\textcolor[HTML]{FF1010}{\texttt{(}}@>
<@\textcolor[HTML]{000000}{\texttt{\ \ \ \ \ \ \ \ \ \ \ \ }}@><@\textcolor[HTML]{EA5110}{\texttt{self}}@><@\textcolor[HTML]{1041FF}{\texttt{.}}@><@\textcolor[HTML]{255CFF}{\texttt{peekToken}}@><@\textcolor[HTML]{FF1010}{\texttt{(}}@><@\textcolor[HTML]{FF1010}{\texttt{)}}@><@\textcolor[HTML]{1041FF}{\texttt{.}}@><@\textcolor[HTML]{000000}{\texttt{start}}@><@\textcolor[HTML]{1041FF}{\texttt{,}}@>
<@\textcolor[HTML]{000000}{\texttt{\ \ \ \ \ \ \ \ \ \ \ \ }}@><@\textcolor[HTML]{EA5110}{\texttt{self}}@><@\textcolor[HTML]{1041FF}{\texttt{.}}@><@\textcolor[HTML]{255CFF}{\texttt{peekToken}}@><@\textcolor[HTML]{FF1010}{\texttt{(}}@><@\textcolor[HTML]{FF1010}{\texttt{)}}@><@\textcolor[HTML]{1041FF}{\texttt{.}}@><@\textcolor[HTML]{000000}{\texttt{end}}@><@\textcolor[HTML]{1041FF}{\texttt{,}}@>
<@\textcolor[HTML]{000000}{\texttt{\ \ \ \ \ \ \ \ \ \ \ \ }}@><@\textcolor[HTML]{418310}{\texttt{"\{s\}\ can't\ be\ at\ the\ beginning\ of\ an\ expression."}}@><@\textcolor[HTML]{1041FF}{\texttt{,}}@>
<@\textcolor[HTML]{000000}{\texttt{\ \ \ \ \ \ \ \ \ \ \ \ }}@><@\textcolor[HTML]{1041FF}{\texttt{.}}@><@\textcolor[HTML]{FF1010}{\texttt{\{}}@><@\textcolor[HTML]{EA5110}{\texttt{token}}@><@\textcolor[HTML]{1041FF}{\texttt{.}}@><@\textcolor[HTML]{255CFF}{\texttt{formatTokenType}}@><@\textcolor[HTML]{FF1010}{\texttt{(}}@><@\textcolor[HTML]{EA5110}{\texttt{self}}@><@\textcolor[HTML]{1041FF}{\texttt{.}}@><@\textcolor[HTML]{255CFF}{\texttt{peekToken}}@><@\textcolor[HTML]{FF1010}{\texttt{(}}@><@\textcolor[HTML]{FF1010}{\texttt{)}}@><@\textcolor[HTML]{1041FF}{\texttt{.}}@><@\textcolor[HTML]{000000}{\texttt{type}}@><@\textcolor[HTML]{FF1010}{\texttt{)}}@><@\textcolor[HTML]{FF1010}{\texttt{\}}}@><@\textcolor[HTML]{1041FF}{\texttt{,}}@>
<@\textcolor[HTML]{000000}{\texttt{\ \ \ \ \ \ \ \ }}@><@\textcolor[HTML]{FF1010}{\texttt{)}}@><@\textcolor[HTML]{1041FF}{\texttt{;}}@>
<@\textcolor[HTML]{000000}{\texttt{\ \ \ \ \ \ \ \ }}@><@\textcolor[HTML]{1010FF}{\textit{\texttt{return}}}@><@\textcolor[HTML]{000000}{\texttt{\ }}@><@\textcolor[HTML]{3010CF}{\textit{\textbf{\texttt{error}}}}@><@\textcolor[HTML]{1041FF}{\texttt{.}}@><@\textcolor[HTML]{EA5110}{\texttt{InvalidPrefix}}@><@\textcolor[HTML]{1041FF}{\texttt{;}}@>
<@\textcolor[HTML]{000000}{\texttt{\ \ \ \ }}@><@\textcolor[HTML]{FF1010}{\texttt{\}}}@><@\textcolor[HTML]{1041FF}{\texttt{;}}@>
<@\textcolor[HTML]{000000}{\texttt{\ \ \ \ }}@><@\textcolor[HTML]{3010CF}{\textit{\textbf{\texttt{var}}}}@><@\textcolor[HTML]{000000}{\texttt{\ }}@><@\textcolor[HTML]{EA5110}{\texttt{left}}@><@\textcolor[HTML]{000000}{\texttt{\ }}@><@\textcolor[HTML]{1041FF}{\texttt{=}}@><@\textcolor[HTML]{000000}{\texttt{\ }}@><@\textcolor[HTML]{911010}{\textit{\texttt{try}}}@><@\textcolor[HTML]{000000}{\texttt{\ }}@><@\textcolor[HTML]{255CFF}{\texttt{n}}@><@\textcolor[HTML]{FF1010}{\texttt{(}}@><@\textcolor[HTML]{EA5110}{\texttt{self}}@><@\textcolor[HTML]{FF1010}{\texttt{)}}@><@\textcolor[HTML]{1041FF}{\texttt{;}}@>
<@\textcolor[HTML]{000000}{\texttt{\ \ \ \ }}@><@\textcolor[HTML]{3010CF}{\textit{\textbf{\texttt{errdefer}}}}@><@\textcolor[HTML]{000000}{\texttt{\ }}@><@\textcolor[HTML]{EA5110}{\texttt{left}}@><@\textcolor[HTML]{1041FF}{\texttt{.}}@><@\textcolor[HTML]{255CFF}{\texttt{deinit}}@><@\textcolor[HTML]{FF1010}{\texttt{(}}@><@\textcolor[HTML]{EA5110}{\texttt{self}}@><@\textcolor[HTML]{1041FF}{\texttt{.}}@><@\textcolor[HTML]{000000}{\texttt{allocator}}@><@\textcolor[HTML]{FF1010}{\texttt{)}}@><@\textcolor[HTML]{1041FF}{\texttt{;}}@>
<@\textcolor[HTML]{000000}{\texttt{\ \ \ \ }}@><@\textcolor[HTML]{7777BB}{\textit{\texttt{//\ led\ stands\ for\ left\ denotation}}}@>
<@\textcolor[HTML]{000000}{\texttt{\ \ \ \ }}@><@\textcolor[HTML]{3010CF}{\textit{\textbf{\texttt{var}}}}@><@\textcolor[HTML]{000000}{\texttt{\ }}@><@\textcolor[HTML]{EA5110}{\texttt{led}}@><@\textcolor[HTML]{000000}{\texttt{\ }}@><@\textcolor[HTML]{1041FF}{\texttt{=}}@><@\textcolor[HTML]{000000}{\texttt{\ }}@><@\textcolor[HTML]{EA5110}{\texttt{self}}@><@\textcolor[HTML]{1041FF}{\texttt{.}}@><@\textcolor[HTML]{255CFF}{\texttt{getLed}}@><@\textcolor[HTML]{FF1010}{\texttt{(}}@><@\textcolor[HTML]{FF1010}{\texttt{)}}@><@\textcolor[HTML]{1041FF}{\texttt{;}}@>
<@\textcolor[HTML]{000000}{\texttt{\ \ \ \ }}@><@\textcolor[HTML]{7777BB}{\textit{\texttt{//\ lbp\ is\ left\ binding\ power}}}@>
<@\textcolor[HTML]{000000}{\texttt{\ \ \ \ }}@><@\textcolor[HTML]{3010CF}{\textit{\textbf{\texttt{var}}}}@><@\textcolor[HTML]{000000}{\texttt{\ }}@><@\textcolor[HTML]{EA5110}{\texttt{c}}@><@\textcolor[HTML]{1041FF}{\texttt{,}}@><@\textcolor[HTML]{000000}{\texttt{\ }}@><@\textcolor[HTML]{3010CF}{\textit{\textbf{\texttt{var}}}}@><@\textcolor[HTML]{000000}{\texttt{\ }}@><@\textcolor[HTML]{EA5110}{\texttt{lbp}}@><@\textcolor[HTML]{000000}{\texttt{\ }}@><@\textcolor[HTML]{1041FF}{\texttt{=}}@><@\textcolor[HTML]{000000}{\texttt{\ }}@><@\textcolor[HTML]{EA5110}{\texttt{led}}@><@\textcolor[HTML]{000000}{\texttt{\ }}@><@\textcolor[HTML]{1010FF}{\texttt{orelse}}@><@\textcolor[HTML]{000000}{\texttt{\ }}@><@\textcolor[HTML]{1010FF}{\textit{\texttt{return}}}@><@\textcolor[HTML]{000000}{\texttt{\ }}@><@\textcolor[HTML]{EA5110}{\texttt{left}}@><@\textcolor[HTML]{1041FF}{\texttt{;}}@>
<@\textcolor[HTML]{000000}{\texttt{\ \ \ \ }}@><@\textcolor[HTML]{3010CF}{\textit{\textbf{\texttt{while}}}}@><@\textcolor[HTML]{000000}{\texttt{\ }}@><@\textcolor[HTML]{FF1010}{\texttt{(}}@><@\textcolor[HTML]{EA5110}{\texttt{rbp}}@><@\textcolor[HTML]{000000}{\texttt{\ }}@><@\textcolor[HTML]{1041FF}{\texttt{<}}@><@\textcolor[HTML]{000000}{\texttt{\ }}@><@\textcolor[HTML]{EA5110}{\texttt{lbp}}@><@\textcolor[HTML]{FF1010}{\texttt{)}}@><@\textcolor[HTML]{000000}{\texttt{\ }}@><@\textcolor[HTML]{FF1010}{\texttt{\{}}@>
<@\textcolor[HTML]{000000}{\texttt{\ \ \ \ \ \ \ \ }}@><@\textcolor[HTML]{EA5110}{\texttt{left}}@><@\textcolor[HTML]{000000}{\texttt{\ }}@><@\textcolor[HTML]{1041FF}{\texttt{=}}@><@\textcolor[HTML]{000000}{\texttt{\ }}@><@\textcolor[HTML]{911010}{\textit{\texttt{try}}}@><@\textcolor[HTML]{000000}{\texttt{\ }}@><@\textcolor[HTML]{255CFF}{\texttt{c}}@><@\textcolor[HTML]{FF1010}{\texttt{(}}@><@\textcolor[HTML]{EA5110}{\texttt{self}}@><@\textcolor[HTML]{1041FF}{\texttt{,}}@><@\textcolor[HTML]{000000}{\texttt{\ }}@><@\textcolor[HTML]{EA5110}{\texttt{left}}@><@\textcolor[HTML]{FF1010}{\texttt{)}}@><@\textcolor[HTML]{1041FF}{\texttt{;}}@>
<@\textcolor[HTML]{000000}{\texttt{\ \ \ \ \ \ \ \ }}@><@\textcolor[HTML]{EA5110}{\texttt{led}}@><@\textcolor[HTML]{000000}{\texttt{\ }}@><@\textcolor[HTML]{1041FF}{\texttt{=}}@><@\textcolor[HTML]{000000}{\texttt{\ }}@><@\textcolor[HTML]{EA5110}{\texttt{self}}@><@\textcolor[HTML]{1041FF}{\texttt{.}}@><@\textcolor[HTML]{255CFF}{\texttt{getLed}}@><@\textcolor[HTML]{FF1010}{\texttt{(}}@><@\textcolor[HTML]{FF1010}{\texttt{)}}@><@\textcolor[HTML]{1041FF}{\texttt{;}}@>
<@\textcolor[HTML]{000000}{\texttt{\ \ \ \ \ \ \ \ }}@><@\textcolor[HTML]{EA5110}{\texttt{c}}@><@\textcolor[HTML]{1041FF}{\texttt{,}}@><@\textcolor[HTML]{000000}{\texttt{\ }}@><@\textcolor[HTML]{EA5110}{\texttt{lbp}}@><@\textcolor[HTML]{000000}{\texttt{\ }}@><@\textcolor[HTML]{1041FF}{\texttt{=}}@><@\textcolor[HTML]{000000}{\texttt{\ }}@><@\textcolor[HTML]{EA5110}{\texttt{led}}@><@\textcolor[HTML]{000000}{\texttt{\ }}@><@\textcolor[HTML]{1010FF}{\texttt{orelse}}@><@\textcolor[HTML]{000000}{\texttt{\ }}@><@\textcolor[HTML]{1010FF}{\textit{\texttt{return}}}@><@\textcolor[HTML]{000000}{\texttt{\ }}@><@\textcolor[HTML]{EA5110}{\texttt{left}}@><@\textcolor[HTML]{1041FF}{\texttt{;}}@>
<@\textcolor[HTML]{000000}{\texttt{\ \ \ \ }}@><@\textcolor[HTML]{FF1010}{\texttt{\}}}@>
<@\textcolor[HTML]{000000}{\texttt{\ \ \ \ }}@><@\textcolor[HTML]{1010FF}{\textit{\texttt{return}}}@><@\textcolor[HTML]{000000}{\texttt{\ }}@><@\textcolor[HTML]{EA5110}{\texttt{left}}@><@\textcolor[HTML]{1041FF}{\texttt{;}}@>
<@\textcolor[HTML]{FF1010}{\texttt{\}}}@>

\end{lstlisting}

The Parser consists of a pretty straight-forward implementation of the Pratt Parsing algorithm
for the expressions as shown in Listing \ref{listing:expressions} and a simple loop for statements.
In this implementation, the lookup table is represented by two functions,
\mintinline{zig}|getNud| and \mintinline{zig}|getLed|.
The problem that there might not be an entry is solved by returning null
in that case.

\subsection{Type Checking/Inference}
The Type Checker is also mostly an implementation close to the Algorithm J as described
in the paper \cite{damasPrincipalTypeschemesFunctional1982}
There is however a significant difference to the original type system:
There exists a special type Number(a), which can be either a Float or an Int
and therefore can be unified by them Both. When unified, the runtime representation
is then Number(Int) or Number(Float).
It is printed similarly to how Type Class constraints are in Haskell.
This means that the type Number(a) is printed as \texttt{Number a => a}.
However, in order to be able to compute free type variables for generalization
faster, levels as described in \cite{EfficientInsightfulGeneralization} are
used. However, they are called depth in the code.

\subsection{The Optimizer}
\begin{lstlisting}[
	% there are many more options of styling, see the official documentation, these are just the defaults I like
	frame=single, % make single-line frame around the verbatim
	framesep=2mm, % put some more spacing between the frame and text
	aboveskip=5mm, % put some more space above the box
	basicstyle={\linespread{0.9}\small\ttfamily}, % use typewriter (monospace) font
	caption={Outline of the Optimizer}, % set the caption text
	captionpos=b, % put the caption at the bottom (b) or top (t) or both (bt)
    label={listing:optimizer}, % label to be referenced via \ref{}
	numbers=left, % line numbers on the left
	numberstyle={\scriptsize\ttfamily\color{black!60}}, % the style for line numbers
	escapeinside={<@}{@>} % between those sequences are commands evaluated
]
<@\textcolor[HTML]{7777BB}{\textit{\texttt{//\ Create\ a\ list\ of\ bound\ variables\ for\ each\ closure}}}@>
<@\textcolor[HTML]{7777BB}{\textit{\texttt{//\ so\ not\ the\ entire\ env\ has\ to\ be\ cloned}}}@>
<@\textcolor[HTML]{3010CF}{\textit{\textbf{\texttt{pub}}}}@><@\textcolor[HTML]{000000}{\texttt{\ }}@><@\textcolor[HTML]{3010CF}{\textit{\textbf{\texttt{const}}}}@><@\textcolor[HTML]{000000}{\texttt{\ }}@><@\textcolor[HTML]{EA5110}{\texttt{GatherBound}}@><@\textcolor[HTML]{000000}{\texttt{\ }}@><@\textcolor[HTML]{1041FF}{\texttt{=}}@><@\textcolor[HTML]{000000}{\texttt{\ }}@><@\textcolor[HTML]{3010CF}{\textit{\textbf{\texttt{struct}}}}@><@\textcolor[HTML]{000000}{\texttt{\ }}@><@\textcolor[HTML]{FF1010}{\texttt{\{}}@>
<@\textcolor[HTML]{000000}{\texttt{\ \ \ \ }}@><@\textcolor[HTML]{3010CF}{\textit{\textbf{\texttt{pub}}}}@><@\textcolor[HTML]{000000}{\texttt{\ }}@><@\textcolor[HTML]{3010CF}{\textbf{\texttt{fn}}}@><@\textcolor[HTML]{000000}{\texttt{\ }}@><@\textcolor[HTML]{255CFF}{\texttt{run}}@><@\textcolor[HTML]{FF1010}{\texttt{(}}@><@\textcolor[HTML]{EA5110}{\texttt{ast}}@><@\textcolor[HTML]{1041FF}{\texttt{:}}@><@\textcolor[HTML]{000000}{\texttt{\ }}@><@\textcolor[HTML]{1041FF}{\texttt{*}}@><@\textcolor[HTML]{EA5110}{\texttt{AST}}@><@\textcolor[HTML]{1041FF}{\texttt{,}}@><@\textcolor[HTML]{000000}{\texttt{\ }}@><@\textcolor[HTML]{EA5110}{\texttt{allocator}}@><@\textcolor[HTML]{1041FF}{\texttt{:}}@><@\textcolor[HTML]{000000}{\texttt{\ }}@><@\textcolor[HTML]{EA5110}{\texttt{std}}@><@\textcolor[HTML]{1041FF}{\texttt{.}}@><@\textcolor[HTML]{000000}{\texttt{mem}}@><@\textcolor[HTML]{1041FF}{\texttt{.}}@><@\textcolor[HTML]{000000}{\texttt{Allocator}}@><@\textcolor[HTML]{FF1010}{\texttt{)}}@><@\textcolor[HTML]{000000}{\texttt{\ }}@><@\textcolor[HTML]{1041FF}{\texttt{!}}@><@\textcolor[HTML]{10909F}{\texttt{void}}@><@\textcolor[HTML]{000000}{\texttt{\ }}@><@\textcolor[HTML]{FF1010}{\texttt{\{}}@>
<@\textcolor[HTML]{000000}{\texttt{\ \ \ \ \ \ \ \ }}@><@\textcolor[HTML]{1010FF}{\texttt{switch}}@><@\textcolor[HTML]{000000}{\texttt{\ }}@><@\textcolor[HTML]{FF1010}{\texttt{(}}@><@\textcolor[HTML]{EA5110}{\texttt{ast}}@><@\textcolor[HTML]{1041FF}{\texttt{.*}}@><@\textcolor[HTML]{FF1010}{\texttt{)}}@><@\textcolor[HTML]{000000}{\texttt{\ }}@><@\textcolor[HTML]{FF1010}{\texttt{\{}}@>
<@\textcolor[HTML]{000000}{\texttt{\ \ \ \ \ \ \ \ \ \ \ \ }}@><@\textcolor[HTML]{1041FF}{\texttt{.}}@><@\textcolor[HTML]{EA5110}{\texttt{intConstant}}@><@\textcolor[HTML]{1041FF}{\texttt{,}}@><@\textcolor[HTML]{000000}{\texttt{\ }}@><@\textcolor[HTML]{1041FF}{\texttt{.}}@><@\textcolor[HTML]{EA5110}{\texttt{floatConstant}}@><@\textcolor[HTML]{1041FF}{\texttt{,}}@><@\textcolor[HTML]{000000}{\texttt{\ }}@><@\textcolor[HTML]{AA1090}{\textit{\texttt{...}}}@><@\textcolor[HTML]{000000}{\texttt{\ }}@><@\textcolor[HTML]{1041FF}{\texttt{=>}}@><@\textcolor[HTML]{000000}{\texttt{\ }}@><@\textcolor[HTML]{FF1010}{\texttt{\{}}@><@\textcolor[HTML]{FF1010}{\texttt{\}}}@><@\textcolor[HTML]{1041FF}{\texttt{,}}@>
<@\textcolor[HTML]{000000}{\texttt{\ \ \ \ \ \ \ \ \ \ \ \ }}@><@\textcolor[HTML]{1041FF}{\texttt{.}}@><@\textcolor[HTML]{EA5110}{\texttt{let}}@><@\textcolor[HTML]{000000}{\texttt{\ }}@><@\textcolor[HTML]{1041FF}{\texttt{=>}}@><@\textcolor[HTML]{000000}{\texttt{\ }}@><@\textcolor[HTML]{FF1010}{\texttt{|}}@><@\textcolor[HTML]{EA5110}{\texttt{let}}@><@\textcolor[HTML]{FF1010}{\texttt{|}}@><@\textcolor[HTML]{000000}{\texttt{\ }}@><@\textcolor[HTML]{FF1010}{\texttt{\{}}@>
<@\textcolor[HTML]{000000}{\texttt{\ \ \ \ \ \ \ \ \ \ \ \ \ \ \ \ }}@><@\textcolor[HTML]{911010}{\textit{\texttt{try}}}@><@\textcolor[HTML]{000000}{\texttt{\ }}@><@\textcolor[HTML]{255CFF}{\texttt{run}}@><@\textcolor[HTML]{FF1010}{\texttt{(}}@><@\textcolor[HTML]{EA5110}{\texttt{let}}@><@\textcolor[HTML]{1041FF}{\texttt{.}}@><@\textcolor[HTML]{000000}{\texttt{in}}@><@\textcolor[HTML]{1041FF}{\texttt{,}}@><@\textcolor[HTML]{000000}{\texttt{\ }}@><@\textcolor[HTML]{EA5110}{\texttt{allocator}}@><@\textcolor[HTML]{FF1010}{\texttt{)}}@><@\textcolor[HTML]{1041FF}{\texttt{;}}@>
<@\textcolor[HTML]{000000}{\texttt{\ \ \ \ \ \ \ \ \ \ \ \ \ \ \ \ }}@><@\textcolor[HTML]{911010}{\textit{\texttt{try}}}@><@\textcolor[HTML]{000000}{\texttt{\ }}@><@\textcolor[HTML]{255CFF}{\texttt{run}}@><@\textcolor[HTML]{FF1010}{\texttt{(}}@><@\textcolor[HTML]{EA5110}{\texttt{let}}@><@\textcolor[HTML]{1041FF}{\texttt{.}}@><@\textcolor[HTML]{000000}{\texttt{be}}@><@\textcolor[HTML]{1041FF}{\texttt{,}}@><@\textcolor[HTML]{000000}{\texttt{\ }}@><@\textcolor[HTML]{EA5110}{\texttt{allocator}}@><@\textcolor[HTML]{FF1010}{\texttt{)}}@><@\textcolor[HTML]{1041FF}{\texttt{;}}@>
<@\textcolor[HTML]{000000}{\texttt{\ \ \ \ \ \ \ \ \ \ \ \ }}@><@\textcolor[HTML]{FF1010}{\texttt{\}}}@><@\textcolor[HTML]{1041FF}{\texttt{,}}@>
<@\textcolor[HTML]{000000}{\texttt{\ \ \ \ \ \ \ \ \ \ \ \ }}@><@\textcolor[HTML]{1041FF}{\texttt{.}}@><@\textcolor[HTML]{EA5110}{\texttt{ifExpr}}@><@\textcolor[HTML]{000000}{\texttt{\ }}@><@\textcolor[HTML]{1041FF}{\texttt{=>}}@><@\textcolor[HTML]{000000}{\texttt{\ }}@><@\textcolor[HTML]{FF1010}{\texttt{|}}@><@\textcolor[HTML]{EA5110}{\texttt{ifExpr}}@><@\textcolor[HTML]{FF1010}{\texttt{|}}@><@\textcolor[HTML]{000000}{\texttt{\ }}@><@\textcolor[HTML]{FF1010}{\texttt{\{}}@>
<@\textcolor[HTML]{000000}{\texttt{\ \ \ \ \ \ \ \ \ \ \ \ \ \ \ \ }}@><@\textcolor[HTML]{911010}{\textit{\texttt{try}}}@><@\textcolor[HTML]{000000}{\texttt{\ }}@><@\textcolor[HTML]{255CFF}{\texttt{run}}@><@\textcolor[HTML]{FF1010}{\texttt{(}}@><@\textcolor[HTML]{EA5110}{\texttt{ifExpr}}@><@\textcolor[HTML]{1041FF}{\texttt{.}}@><@\textcolor[HTML]{000000}{\texttt{predicate}}@><@\textcolor[HTML]{1041FF}{\texttt{,}}@><@\textcolor[HTML]{000000}{\texttt{\ }}@><@\textcolor[HTML]{EA5110}{\texttt{allocator}}@><@\textcolor[HTML]{FF1010}{\texttt{)}}@><@\textcolor[HTML]{1041FF}{\texttt{;}}@>
<@\textcolor[HTML]{000000}{\texttt{\ \ \ \ \ \ \ \ \ \ \ \ \ \ \ \ }}@><@\textcolor[HTML]{911010}{\textit{\texttt{try}}}@><@\textcolor[HTML]{000000}{\texttt{\ }}@><@\textcolor[HTML]{255CFF}{\texttt{run}}@><@\textcolor[HTML]{FF1010}{\texttt{(}}@><@\textcolor[HTML]{EA5110}{\texttt{ifExpr}}@><@\textcolor[HTML]{1041FF}{\texttt{.}}@><@\textcolor[HTML]{000000}{\texttt{thenExpr}}@><@\textcolor[HTML]{1041FF}{\texttt{,}}@><@\textcolor[HTML]{000000}{\texttt{\ }}@><@\textcolor[HTML]{EA5110}{\texttt{allocator}}@><@\textcolor[HTML]{FF1010}{\texttt{)}}@><@\textcolor[HTML]{1041FF}{\texttt{;}}@>
<@\textcolor[HTML]{000000}{\texttt{\ \ \ \ \ \ \ \ \ \ \ \ \ \ \ \ }}@><@\textcolor[HTML]{911010}{\textit{\texttt{try}}}@><@\textcolor[HTML]{000000}{\texttt{\ }}@><@\textcolor[HTML]{255CFF}{\texttt{run}}@><@\textcolor[HTML]{FF1010}{\texttt{(}}@><@\textcolor[HTML]{EA5110}{\texttt{ifExpr}}@><@\textcolor[HTML]{1041FF}{\texttt{.}}@><@\textcolor[HTML]{000000}{\texttt{elseExpr}}@><@\textcolor[HTML]{1041FF}{\texttt{,}}@><@\textcolor[HTML]{000000}{\texttt{\ }}@><@\textcolor[HTML]{EA5110}{\texttt{allocator}}@><@\textcolor[HTML]{FF1010}{\texttt{)}}@><@\textcolor[HTML]{1041FF}{\texttt{;}}@>
<@\textcolor[HTML]{000000}{\texttt{\ \ \ \ \ \ \ \ \ \ \ \ }}@><@\textcolor[HTML]{FF1010}{\texttt{\}}}@><@\textcolor[HTML]{1041FF}{\texttt{,}}@>
<@\textcolor[HTML]{000000}{\texttt{\ \ \ \ \ \ \ \ \ \ \ \ }}@><@\textcolor[HTML]{AA1090}{\textit{\texttt{...}}}@>
<@\textcolor[HTML]{000000}{\texttt{\ \ \ \ \ \ \ \ }}@><@\textcolor[HTML]{FF1010}{\texttt{\}}}@>
<@\textcolor[HTML]{000000}{\texttt{\ \ \ \ }}@><@\textcolor[HTML]{FF1010}{\texttt{\}}}@>
<@\textcolor[HTML]{000000}{\texttt{\ \ \ \ }}@><@\textcolor[HTML]{AA1090}{\textit{\texttt{...}}}@>
<@\textcolor[HTML]{FF1010}{\texttt{\}}}@><@\textcolor[HTML]{1041FF}{\texttt{;}}@>

<@\textcolor[HTML]{7777BB}{\textit{\texttt{//\ Convert\ consecutive\ lambdas\ or\ calls\ into\ one\ object}}}@>
<@\textcolor[HTML]{3010CF}{\textit{\textbf{\texttt{pub}}}}@><@\textcolor[HTML]{000000}{\texttt{\ }}@><@\textcolor[HTML]{3010CF}{\textit{\textbf{\texttt{const}}}}@><@\textcolor[HTML]{000000}{\texttt{\ }}@><@\textcolor[HTML]{EA5110}{\texttt{CombineCallsAndLambdas}}@><@\textcolor[HTML]{000000}{\texttt{\ }}@><@\textcolor[HTML]{1041FF}{\texttt{=}}@><@\textcolor[HTML]{000000}{\texttt{\ }}@><@\textcolor[HTML]{3010CF}{\textit{\textbf{\texttt{struct}}}}@><@\textcolor[HTML]{000000}{\texttt{\ }}@><@\textcolor[HTML]{FF1010}{\texttt{\{}}@>
<@\textcolor[HTML]{000000}{\texttt{\ \ \ \ }}@><@\textcolor[HTML]{3010CF}{\textit{\textbf{\texttt{pub}}}}@><@\textcolor[HTML]{000000}{\texttt{\ }}@><@\textcolor[HTML]{3010CF}{\textbf{\texttt{fn}}}@><@\textcolor[HTML]{000000}{\texttt{\ }}@><@\textcolor[HTML]{255CFF}{\texttt{run}}@><@\textcolor[HTML]{FF1010}{\texttt{(}}@><@\textcolor[HTML]{EA5110}{\texttt{ast}}@><@\textcolor[HTML]{1041FF}{\texttt{:}}@><@\textcolor[HTML]{000000}{\texttt{\ }}@><@\textcolor[HTML]{1041FF}{\texttt{*}}@><@\textcolor[HTML]{EA5110}{\texttt{AST}}@><@\textcolor[HTML]{1041FF}{\texttt{,}}@><@\textcolor[HTML]{000000}{\texttt{\ }}@><@\textcolor[HTML]{EA5110}{\texttt{allocator}}@><@\textcolor[HTML]{1041FF}{\texttt{:}}@><@\textcolor[HTML]{000000}{\texttt{\ }}@><@\textcolor[HTML]{EA5110}{\texttt{std}}@><@\textcolor[HTML]{1041FF}{\texttt{.}}@><@\textcolor[HTML]{000000}{\texttt{mem}}@><@\textcolor[HTML]{1041FF}{\texttt{.}}@><@\textcolor[HTML]{000000}{\texttt{Allocator}}@><@\textcolor[HTML]{FF1010}{\texttt{)}}@><@\textcolor[HTML]{000000}{\texttt{\ }}@><@\textcolor[HTML]{1041FF}{\texttt{!}}@><@\textcolor[HTML]{10909F}{\texttt{void}}@><@\textcolor[HTML]{000000}{\texttt{\ }}@><@\textcolor[HTML]{FF1010}{\texttt{\{}}@>
<@\textcolor[HTML]{000000}{\texttt{\ \ \ \ \ \ \ \ }}@><@\textcolor[HTML]{AA1090}{\textit{\texttt{...}}}@>
<@\textcolor[HTML]{000000}{\texttt{\ \ \ \ }}@><@\textcolor[HTML]{FF1010}{\texttt{\}}}@>
<@\textcolor[HTML]{000000}{\texttt{\ \ \ \ }}@><@\textcolor[HTML]{AA1090}{\textit{\texttt{...}}}@>
<@\textcolor[HTML]{FF1010}{\texttt{\}}}@><@\textcolor[HTML]{1041FF}{\texttt{;}}@>

<@\textcolor[HTML]{7777BB}{\textit{\texttt{//\ Run\ all\ optimizations\ consecutively\ on\ a\ statement}}}@>
<@\textcolor[HTML]{3010CF}{\textit{\textbf{\texttt{pub}}}}@><@\textcolor[HTML]{000000}{\texttt{\ }}@><@\textcolor[HTML]{3010CF}{\textbf{\texttt{fn}}}@><@\textcolor[HTML]{000000}{\texttt{\ }}@><@\textcolor[HTML]{255CFF}{\texttt{optimizeStatement}}@><@\textcolor[HTML]{FF1010}{\texttt{(}}@>
<@\textcolor[HTML]{000000}{\texttt{\ \ \ \ }}@><@\textcolor[HTML]{EA5110}{\texttt{statement}}@><@\textcolor[HTML]{1041FF}{\texttt{:}}@><@\textcolor[HTML]{000000}{\texttt{\ }}@><@\textcolor[HTML]{1041FF}{\texttt{*}}@><@\textcolor[HTML]{EA5110}{\texttt{Statement}}@><@\textcolor[HTML]{1041FF}{\texttt{,}}@>
<@\textcolor[HTML]{000000}{\texttt{\ \ \ \ }}@><@\textcolor[HTML]{EA5110}{\texttt{allocator}}@><@\textcolor[HTML]{1041FF}{\texttt{:}}@><@\textcolor[HTML]{000000}{\texttt{\ }}@><@\textcolor[HTML]{EA5110}{\texttt{std}}@><@\textcolor[HTML]{1041FF}{\texttt{.}}@><@\textcolor[HTML]{000000}{\texttt{mem}}@><@\textcolor[HTML]{1041FF}{\texttt{.}}@><@\textcolor[HTML]{000000}{\texttt{Allocator}}@><@\textcolor[HTML]{1041FF}{\texttt{,}}@>
<@\textcolor[HTML]{000000}{\texttt{\ \ \ \ }}@><@\textcolor[HTML]{EA5110}{\texttt{interpreted}}@><@\textcolor[HTML]{1041FF}{\texttt{:}}@><@\textcolor[HTML]{000000}{\texttt{\ }}@><@\textcolor[HTML]{10909F}{\texttt{bool}}@>
<@\textcolor[HTML]{FF1010}{\texttt{)}}@><@\textcolor[HTML]{000000}{\texttt{\ }}@><@\textcolor[HTML]{1041FF}{\texttt{!}}@><@\textcolor[HTML]{10909F}{\texttt{void}}@><@\textcolor[HTML]{000000}{\texttt{\ }}@><@\textcolor[HTML]{FF1010}{\texttt{\{}}@>
<@\textcolor[HTML]{000000}{\texttt{\ \ \ \ }}@><@\textcolor[HTML]{AA1090}{\textit{\texttt{...}}}@>
<@\textcolor[HTML]{FF1010}{\texttt{\}}}@>

\end{lstlisting}

The Optimizer works slightly different depending on whether
the code is interpreted or compiled. In the case that it is interpreted
the optimizer also combines lambdas into larger objects,
while the compiler doesn't need this. However, in both cases
a list of bound variables is gathered per function definition,
since all functions are closures.
The outline of the optimizer is shown in Listing \ref{listing:optimizer}.

\subsection{The Interpreter}
The Interpreter is a simple tree-walking Interpreter,
meaning it directly uses the AST to run the code.
The most important optimization it employs is proper tail call elimination,
meaning it doesn't make another stack frame if the function call is the last
thing a function does, making recursion feasible through
tail recursion. This is done by creating a simple form of trampolining,
where the evaluation of an expression returns the next thing to
do, which is then repeatedly evaluated in a loop until the final result
is reached.

\subsection{The Compiler}
The Compiler converts the AST into LLVM IR using the C API through
Zig's FFI. Here, little is done to optimize the IR created apart
from recursive calls being done directly instead of through an indirect
call, as all other function calls are done.
Closures are represented in the IR as a struct consisting of two pointers:
The first being the function pointer, the second being a pointer to the heap
where all bound variables are stored.
Then clang is called on the IR, linking it with a very small library of builtin
functions with the highest optimization level of -O3.

\section{Discussion}

\section{Conclusions}

\newpage
\bibliographystyle{IEEEtran}
\bibliography{bibliography}

\listoflistings

\appendix
\section{Source Code}
The source code for this project can be found at: \href{https://github.com/enmiligi/matura-project}{GitHub}

\end{document}